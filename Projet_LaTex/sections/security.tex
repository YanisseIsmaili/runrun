\section{Sécurité de l'Application}

\subsection{Architecture de Sécurité}

\begin{figure}[H]
\centering
\begin{tikzpicture}[node distance=2cm]
    % Client
    \node[rectangle, draw, fill=green!20, text width=2.5cm, text centered] (client) {
        \textbf{Mobile App}\\
        \small Token Storage\\
        \small Input Validation
    };
    
    % Internet
    \node[cloud, draw, fill=gray!20, text width=2cm, text centered, right of=client, xshift=2cm] (internet) {
        \textbf{Internet}\\
        \small HTTPS/TLS
    };
    
    % Load Balancer
    \node[rectangle, draw, fill=blue!20, text width=2.5cm, text centered, right of=internet, xshift=2cm] (lb) {
        \textbf{Load Balancer}\\
        \small SSL Termination\\
        \small Rate Limiting
    };
    
    % API
    \node[rectangle, draw, fill=blue!30, text width=2.5cm, text centered, below of=lb] (api) {
        \textbf{API Flask}\\
        \small JWT Validation\\
        \small CORS Policy
    };
    
    % Database
    \node[rectangle, draw, fill=orange!20, text width=2.5cm, text centered, below of=api] (db) {
        \textbf{MySQL}\\
        \small Encrypted Storage\\
        \small Access Control
    };
    
    % Arrows
    \draw[->, thick] (client) -- (internet);
    \draw[->, thick] (internet) -- (lb);
    \draw[->, thick] (lb) -- (api);
    \draw[->, thick] (api) -- (db);
    
    % Security labels
    \node[text=red, above of=client, yshift=-0.5cm] {\small Validation};
    \node[text=red, above of=internet, yshift=-0.5cm] {\small Chiffrement};
    \node[text=red, above of=lb, yshift=-0.5cm] {\small Protection};
    \node[text=red, right of=api, xshift=1cm] {\small Autorisation};
    \node[text=red, right of=db, xshift=1cm] {\small Persistance};
    
\end{tikzpicture}
\caption{Couches de Sécurité Running App}
\end{figure}

\subsection{Authentification et Autorisation}

\subsubsection{JSON Web Tokens (JWT)}

\paragraph{Implémentation JWT}
\begin{lstlisting}[language=python]
# Configuration JWT sécurisée
app.config['JWT_SECRET_KEY'] = os.environ.get('JWT_SECRET_KEY')
app.config['JWT_ACCESS_TOKEN_EXPIRES'] = timedelta(hours=1)
app.config['JWT_REFRESH_TOKEN_EXPIRES'] = timedelta(days=30)
app.config['JWT_BLACKLIST_ENABLED'] = True
app.config['JWT_BLACKLIST_TOKEN_CHECKS'] = ['access', 'refresh']

# Génération de token sécurisé
@auth_bp.route('/login', methods=['POST'])
def login():
    # Validation des credentials
    user = authenticate_user(email, password)
    if user and user.verify_password(password):
        access_token = create_access_token(
            identity=user.id,
            additional_claims={'role': user.role}
        )
        refresh_token = create_refresh_token(identity=user.id)
        return jsonify({
            'access_token': access_token,
            'refresh_token': refresh_token
        })
\end{lstlisting}

\paragraph{Avantages JWT}
\begin{itemize}
    \item \textbf{Stateless} : Pas de stockage serveur requis
    \item \textbf{Scalabilité} : Distribution sur plusieurs serveurs
    \item \textbf{Performance} : Validation locale sans base de données
    \item \textbf{Standards} : RFC 7519, interopérabilité
\end{itemize}

\paragraph{Gestion des Tokens Côté Client}
\begin{lstlisting}[language=javascript]
// Stockage sécurisé des tokens
const storeTokens = async (accessToken, refreshToken) => {
    await AsyncStorage.setItem('access_token', accessToken);
    await AsyncStorage.setItem('refresh_token', refreshToken);
};

// Intercepteur pour renouvellement automatique
axios.interceptors.response.use(
    response => response,
    async error => {
        if (error.response?.status === 401) {
            const refreshToken = await AsyncStorage.getItem('refresh_token');
            if (refreshToken) {
                try {
                    const response = await refreshAccessToken(refreshToken);
                    await storeTokens(response.access_token, refreshToken);
                    // Retry original request
                    return axios.request(error.config);
                } catch (refreshError) {
                    await logout(); // Force logout
                }
            }
        }
        return Promise.reject(error);
    }
);
\end{lstlisting}

\subsubsection{Gestion des Mots de Passe}

\paragraph{Hachage Sécurisé}
\begin{lstlisting}[language=python]
from werkzeug.security import generate_password_hash, check_password_hash

class User(db.Model):
    password_hash = db.Column(db.String(128), nullable=False)
    
    @property
    def password(self):
        raise AttributeError('password is not a readable attribute')
    
    @password.setter
    def password(self, password):
        # Validation de la complexité
        if not self._validate_password_strength(password):
            raise ValueError('Password does not meet requirements')
        # Hachage avec salt automatique
        self.password_hash = generate_password_hash(
            password, 
            method='pbkdf2:sha256:260000'  # 260k iterations
        )
    
    def verify_password(self, password):
        return check_password_hash(self.password_hash, password)
    
    def _validate_password_strength(self, password):
        """Validation de la force du mot de passe"""
        import re
        return (
            len(password) >= 8 and
            re.search(r'[A-Z]', password) and    # Majuscule
            re.search(r'[a-z]', password) and    # Minuscule  
            re.search(r'[0-9]', password) and    # Chiffre
            re.search(r'[!@#$%^&*]', password)   # Caractère spécial
        )
\end{lstlisting}

\paragraph{Politique de Mots de Passe}
\begin{itemize}
    \item \textbf{Longueur minimum} : 8 caractères
    \item \textbf{Complexité} : Majuscules, minuscules, chiffres, caractères spéciaux
    \item \textbf{Historique} : Interdiction des 5 derniers mots de passe
    \item \textbf{Expiration} : Recommandation de changement tous les 6 mois
    \item \textbf{Tentatives} : Blocage après 5 tentatives échouées
\end{itemize}

\subsection{Protection contre les Attaques}

\subsubsection{CORS (Cross-Origin Resource Sharing)}

\begin{lstlisting}[language=python]
# Configuration CORS sécurisée
CORS(app, resources={
    r"/api/*": {
        "origins": [
            "https://runningapp.com",
            "https://app.runningapp.com",
            "http://localhost:3000"  # Dev uniquement
        ],
        "methods": ["GET", "POST", "PUT", "DELETE"],
        "allow_headers": ["Content-Type", "Authorization"],
        "supports_credentials": True,
        "max_age": 600  # Cache preflight 10 minutes
    }
})

# Headers de sécurité supplémentaires
@app.after_request
def security_headers(response):
    response.headers['X-Content-Type-Options'] = 'nosniff'
    response.headers['X-Frame-Options'] = 'DENY'
    response.headers['X-XSS-Protection'] = '1; mode=block'
    response.headers['Strict-Transport-Security'] = \
        'max-age=31536000; includeSubDomains'
    return response
\end{lstlisting}

\subsubsection{Rate Limiting}

\begin{lstlisting}[language=python]
from flask_limiter import Limiter
from flask_limiter.util import get_remote_address

limiter = Limiter(
    app,
    key_func=get_remote_address,
    default_limits=["1000 per hour", "100 per minute"]
)

# Limites spécifiques par endpoint
@auth_bp.route('/login', methods=['POST'])
@limiter.limit("5 per minute")  # Protection force brute
def login():
    pass

@runs_bp.route('', methods=['POST'])
@limiter.limit("10 per minute")  # Limite création courses
@jwt_required()
def create_run():
    pass
\end{lstlisting}

\subsubsection{Validation et Sanitization}

\begin{lstlisting}[language=python]
from marshmallow import Schema, fields, validate

class RunSchema(Schema):
    start_time = fields.DateTime(required=True)
    end_time = fields.DateTime()
    duration = fields.Integer(validate=validate.Range(min=1, max=86400))
    distance = fields.Float(validate=validate.Range(min=0.01, max=500000))
    route_data = fields.List(fields.Dict())
    
    class Meta:
        # Champs inconnus rejetés
        unknown = marshmallow.EXCLUDE

# Validation automatique
@runs_bp.route('', methods=['POST'])
@jwt_required()
def create_run():
    schema = RunSchema()
    try:
        data = schema.load(request.json)
    except ValidationError as err:
        return jsonify({'errors': err.messages}), 400
\end{lstlisting}

\subsection{Sécurité des Données}

\subsubsection{Chiffrement en Transit}

\begin{itemize}
    \item \textbf{HTTPS obligatoire} : TLS 1.3 minimum
    \item \textbf{Certificate Pinning} : Validation stricte des certificats
    \item \textbf{HSTS} : Strict Transport Security activé
    \item \textbf{Perfect Forward Secrecy} : Rotation des clés de session
\end{itemize}

\begin{lstlisting}[language=javascript]
// Certificate Pinning React Native
import { NetworkingModule } from 'react-native';

const certificatePinner = {
    hostname: 'api.runningapp.com',
    publicKeyHash: 'sha256/ABC123...'  // Hash de la clé publique
};

// Configuration Axios avec pinning
const apiClient = axios.create({
    baseURL: 'https://api.runningapp.com',
    httpsAgent: certificatePinner
});
\end{lstlisting}

\subsubsection{Chiffrement au Repos}

\begin{lstlisting}[language=sql]
-- Chiffrement base de données MySQL
CREATE TABLE users (
    id INT PRIMARY KEY,
    username VARCHAR(64),
    email VARCHAR(120),
    -- Données sensibles chiffrées
    personal_data JSON 
) ENCRYPTION='Y' TABLESPACE=encrypted_ts;

-- Configuration MySQL avec chiffrement
[mysqld]
early-plugin-load=keyring_file.so
keyring_file_data=/var/lib/mysql-keyring/keyring
innodb_undo_log_encrypt=ON
innodb_redo_log_encrypt=ON
binlog_encryption=ON
\end{lstlisting}

\subsubsection{Protection des Données Personnelles (RGPD)}

\paragraph{Mesures Techniques}
\begin{itemize}
    \item \textbf{Pseudonymisation} : Identifiants non réversibles
    \item \textbf{Minimisation} : Collecte limitée aux données nécessaires
    \item \textbf{Rétention} : Suppression automatique après 5 ans
    \item \textbf{Portabilité} : Export des données utilisateur
    \item \textbf{Droit à l'oubli} : Suppression complète sur demande
\end{itemize}

\begin{lstlisting}[language=python]
# Service de gestion RGPD
class GDPRService:
    def export_user_data(self, user_id):
        """Export de toutes les données utilisateur"""
        user = User.query.get(user_id)
        runs = Run.query.filter_by(user_id=user_id).all()
        
        return {
            'personal_data': user.to_dict(include_sensitive=True),
            'runs_data': [run.to_dict() for run in runs],
            'export_date': datetime.utcnow().isoformat()
        }
    
    def anonymize_user_data(self, user_id):
        """Anonymisation des données (alternative à la suppression)"""
        user = User.query.get(user_id)
        user.email = f"anonymized_{user_id}@deleted.local"
        user.username = f"user_{user_id}_deleted"
        user.first_name = "ANONYMIZED"
        user.last_name = "ANONYMIZED"
        db.session.commit()
    
    def delete_user_completely(self, user_id):
        """Suppression complète (cascade)"""
        user = User.query.get(user_id)
        db.session.delete(user)  # Cascade delete runs
        db.session.commit()
\end{lstlisting}

\subsection{Logging et Monitoring de Sécurité}

\subsubsection{Logs de Sécurité}

\begin{lstlisting}[language=python]
import logging
from datetime import datetime

# Configuration logging sécurisé
security_logger = logging.getLogger('security')
security_handler = logging.FileHandler('/var/log/running-app/security.log')
security_formatter = logging.Formatter(
    '%(asctime)s - %(levelname)s - %(message)s - IP:%(remote_addr)s'
)
security_handler.setFormatter(security_formatter)
security_logger.addHandler(security_handler)

# Middleware de logging sécurisé
@app.before_request
def log_security_events():
    # Log des tentatives de connexion
    if request.endpoint == 'auth.login':
        security_logger.info(
            f"Login attempt for {request.json.get('email', 'unknown')}",
            extra={'remote_addr': request.remote_addr}
        )
    
    # Log des accès admin
    if request.path.startswith('/api/admin'):
        security_logger.warning(
            f"Admin access to {request.path}",
            extra={'remote_addr': request.remote_addr}
        )

# Log des événements critiques
def log_security_event(event_type, user_id=None, details=None):
    security_logger.critical(
        f"SECURITY_EVENT: {event_type}",
        extra={
            'user_id': user_id,
            'details': details,
            'timestamp': datetime.utcnow().isoformat()
        }
    )
\end{lstlisting}

\subsubsection{Monitoring et Alertes}

\begin{table}[H]
\centering
\begin{tabular}{|l|l|l|l|}
\hline
\textbf{Événement} & \textbf{Seuil} & \textbf{Action} & \textbf{Gravité} \\
\hline
Tentatives de connexion & 5/min/IP & Blocage IP 15min & Moyenne \\
Échecs JWT & 10/min & Alerte admin & Élevée \\
Accès admin & Chaque accès & Log détaillé & Info \\
Erreurs 500 & 10/min & Alerte technique & Élevée \\
Requêtes suspectes & Pattern ML & Analyse approfondie & Variable \\
\hline
\end{tabular}
\caption{Règles de Monitoring Sécurisé}
\end{table}

\subsection{Tests de Sécurité}

\subsubsection{Tests Automatisés}

\begin{lstlisting}[language=python]
# Tests de sécurité automatisés
class SecurityTestCase(unittest.TestCase):
    
    def test_sql_injection_protection(self):
        """Test protection injection SQL"""
        malicious_payload = "'; DROP TABLE users; --"
        response = self.client.post('/api/auth/login', json={
            'email': malicious_payload,
            'password': 'test'
        })
        # Vérifier que l'injection échoue
        self.assertEqual(response.status_code, 400)
    
    def test_xss_protection(self):
        """Test protection XSS"""
        xss_payload = "<script>alert('xss')</script>"
        response = self.client.put('/api/users/profile', 
            headers={'Authorization': f'Bearer {self.token}'},
            json={'first_name': xss_payload}
        )
        # Vérifier sanitization
        user = User.query.get(self.user_id)
        self.assertNotIn('<script>', user.first_name)
    
    def test_jwt_expiration(self):
        """Test expiration des tokens JWT"""
        with freeze_time("2024-01-01 12:00:00"):
            token = create_access_token(identity=1)
        
        with freeze_time("2024-01-01 14:00:00"):  # +2h
            response = self.client.get('/api/users/profile',
                headers={'Authorization': f'Bearer {token}'})
            self.assertEqual(response.status_code, 401)
\end{lstlisting}

\subsubsection{Audit de Sécurité}

\paragraph{Processus d'Audit}
\begin{enumerate}
    \item \textbf{Scan automatique} : OWASP ZAP, Nessus
    \item \textbf{Review code} : Analyse statique avec Bandit (Python)
    \item \textbf{Penetration testing} : Tests d'intrusion manuels
    \item \textbf{Audit de configuration} : Serveurs et base de données
    \item \textbf{Formation équipe} : Sensibilisation sécurité développeurs
\end{enumerate}

\paragraph{Checklist OWASP Top 10}
\begin{itemize}
    \item ✓ A01 - Broken Access Control : JWT + RBAC
    \item ✓ A02 - Cryptographic Failures : TLS + Hachage
    \item ✓ A03 - Injection : Validation + ORM
    \item ✓ A04 - Insecure Design : Architecture sécurisée
    \item ✓ A05 - Security Misconfiguration : Hardening
    \item ✓ A06 - Vulnerable Components : Audit dépendances
    \item ✓ A07 - Authentication Failures : Politique forte
    \item ✓ A08 - Software Integrity Failures : CI/CD sécurisé
    \item ✓ A09 - Logging Failures : Monitoring complet
    \item ✓ A10 - SSRF : Validation URLs externes
\end{itemize}

\subsection{Plan de Réponse aux Incidents}

\subsubsection{Procédure d'Urgence}

\begin{enumerate}
    \item \textbf{Détection} : Alertes automatiques + monitoring
    \item \textbf{Évaluation} : Classification gravité (1-4)
    \item \textbf{Isolation} : Mise en quarantaine système compromis
    \item \textbf{Investigation} : Analyse logs + forensique
    \item \textbf{Correction} : Patch + mise à jour sécurité
    \item \textbf{Communication} : Notification utilisateurs si nécessaire
    \item \textbf{Post-mortem} : Analyse et amélioration processus
\end{enumerate}

\subsubsection{Contacts d'Urgence}

\begin{table}[H]
\centering
\begin{tabular}{|l|l|l|}
\hline
\textbf{Rôle} & \textbf{Responsabilité} & \textbf{Délai} \\
\hline
Security Officer & Coordination incident & 15 minutes \\
Lead Developer & Analyse technique & 30 minutes \\
DevOps Engineer & Infrastructure & 30 minutes \\
Legal Team & Conformité RGPD & 2 heures \\
Communication & Utilisateurs & 4 heures \\
\hline
\end{tabular}
\caption{Équipe de Réponse aux Incidents}
\end{table}