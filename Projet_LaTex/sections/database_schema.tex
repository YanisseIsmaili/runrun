\section{Annexe B - Schéma Complet de la Base de Données}

\subsection{Script de Création des Tables}

\subsubsection{Table Users}

\begin{lstlisting}[language=sql]
CREATE TABLE users (
    id INTEGER AUTO_INCREMENT PRIMARY KEY,
    username VARCHAR(64) NOT NULL UNIQUE,
    email VARCHAR(120) NOT NULL UNIQUE,
    password_hash VARCHAR(128) NOT NULL,
    first_name VARCHAR(64),
    last_name VARCHAR(64),
    date_of_birth DATE,
    height FLOAT COMMENT 'Taille en centimètres',
    weight FLOAT COMMENT 'Poids en kilogrammes',
    profile_picture VARCHAR(255),
    is_admin BOOLEAN DEFAULT FALSE,
    created_at DATETIME DEFAULT CURRENT_TIMESTAMP,
    updated_at DATETIME DEFAULT CURRENT_TIMESTAMP ON UPDATE CURRENT_TIMESTAMP,
    
    -- Index pour les performances
    INDEX idx_users_email (email),
    INDEX idx_users_username (username),
    INDEX idx_users_admin (is_admin),
    INDEX idx_users_created (created_at),
    
    -- Contraintes
    CONSTRAINT chk_users_height CHECK (height IS NULL OR (height >= 100 AND height <= 250)),
    CONSTRAINT chk_users_weight CHECK (weight IS NULL OR (weight >= 30 AND weight <= 200)),
    CONSTRAINT chk_users_email_format CHECK (email REGEXP '^[A-Za-z0-9._%+-]+@[A-Za-z0-9.-]+\.[A-Za-z]{2,}$')
) ENGINE=InnoDB DEFAULT CHARSET=utf8mb4 COLLATE=utf8mb4_unicode_ci;
\end{lstlisting}

\subsubsection{Table Runs}

\begin{lstlisting}[language=sql]
CREATE TABLE runs (
    id INTEGER AUTO_INCREMENT PRIMARY KEY,
    user_id INTEGER NOT NULL,
    start_time DATETIME NOT NULL,
    end_time DATETIME,
    duration INTEGER COMMENT 'Durée en secondes',
    distance FLOAT COMMENT 'Distance en mètres',
    avg_speed FLOAT COMMENT 'Vitesse moyenne en m/s',
    max_speed FLOAT COMMENT 'Vitesse maximale en m/s',
    calories INTEGER COMMENT 'Calories brûlées (estimation)',
    route_data JSON COMMENT 'Coordonnées GPS au format JSON',
    created_at DATETIME DEFAULT CURRENT_TIMESTAMP,
    updated_at DATETIME DEFAULT CURRENT_TIMESTAMP ON UPDATE CURRENT_TIMESTAMP,
    
    -- Clés étrangères
    FOREIGN KEY (user_id) REFERENCES users(id) ON DELETE CASCADE ON UPDATE CASCADE,
    
    -- Index pour les performances
    INDEX idx_runs_user_id (user_id),
    INDEX idx_runs_start_time (start_time),
    INDEX idx_runs_distance (distance),
    INDEX idx_runs_user_date (user_id, start_time DESC),
    INDEX idx_runs_timerange (start_time, end_time),
    
    -- Contraintes métier
    CONSTRAINT chk_runs_duration CHECK (duration IS NULL OR duration > 0),
    CONSTRAINT chk_runs_distance CHECK (distance IS NULL OR distance >= 0),
    CONSTRAINT chk_runs_speeds CHECK (
        (avg_speed IS NULL OR avg_speed >= 0) AND 
        (max_speed IS NULL OR max_speed >= 0) AND
        (avg_speed IS NULL OR max_speed IS NULL OR max_speed >= avg_speed)
    ),
    CONSTRAINT chk_runs_calories CHECK (calories IS NULL OR calories >= 0),
    CONSTRAINT chk_runs_time_order CHECK (end_time IS NULL OR end_time >= start_time)
) ENGINE=InnoDB DEFAULT CHARSET=utf8mb4 COLLATE=utf8mb4_unicode_ci;
\end{lstlisting}

\subsubsection{Table Admin\_Logs}

\begin{lstlisting}[language=sql]
CREATE TABLE admin_logs (
    id INTEGER AUTO_INCREMENT PRIMARY KEY,
    admin_id INTEGER NOT NULL,
    action VARCHAR(50) NOT NULL,
    resource_type VARCHAR(50) NOT NULL,
    resource_id INTEGER,
    details TEXT,
    ip_address VARCHAR(45) COMMENT 'Support IPv4 et IPv6',
    user_agent VARCHAR(255),
    created_at DATETIME DEFAULT CURRENT_TIMESTAMP,
    
    -- Clés étrangères
    FOREIGN KEY (admin_id) REFERENCES users(id) ON DELETE CASCADE ON UPDATE CASCADE,
    
    -- Index pour les performances
    INDEX idx_admin_logs_admin_id (admin_id),
    INDEX idx_admin_logs_action (action),
    INDEX idx_admin_logs_resource (resource_type, resource_id),
    INDEX idx_admin_logs_date (created_at),
    INDEX idx_admin_logs_admin_date (admin_id, created_at DESC),
    
    -- Contraintes
    CONSTRAINT chk_admin_logs_action CHECK (action IN (
        'CREATE', 'UPDATE', 'DELETE', 'VIEW', 'LOGIN', 'LOGOUT', 
        'EXPORT', 'IMPORT', 'BACKUP', 'RESTORE', 'CONFIG_CHANGE'
    )),
    CONSTRAINT chk_admin_logs_resource CHECK (resource_type IN (
        'USER', 'RUN', 'ADMIN_LOG', 'SYSTEM', 'DATABASE', 'CONFIG'
    ))
) ENGINE=InnoDB DEFAULT CHARSET=utf8mb4 COLLATE=utf8mb4_unicode_ci;
\end{lstlisting}

\subsection{Vues Matérialisées et Procédures}

\subsubsection{Vue User\_Stats}

\begin{lstlisting}[language=sql]
-- Vue pour les statistiques utilisateur optimisées
CREATE VIEW user_stats AS
SELECT 
    u.id as user_id,
    u.username,
    u.email,
    COUNT(r.id) as total_runs,
    COALESCE(SUM(r.distance), 0) as total_distance,
    COALESCE(SUM(r.duration), 0) as total_duration,
    COALESCE(AVG(r.avg_speed), 0) as overall_avg_speed,
    COALESCE(MAX(r.max_speed), 0) as personal_best_speed,
    COALESCE(SUM(r.calories), 0) as total_calories,
    COALESCE(MAX(r.distance), 0) as longest_run,
    COALESCE(MIN(CASE WHEN r.distance > 0 AND r.duration > 0 
                     THEN (r.duration / 60) / (r.distance / 1000) 
                     ELSE NULL END), 0) as best_pace_min_per_km,
    -- Statistiques du mois en cours
    COUNT(CASE WHEN r.start_time >= DATE_FORMAT(CURDATE(), '%Y-%m-01') 
               THEN 1 END) as current_month_runs,
    COALESCE(SUM(CASE WHEN r.start_time >= DATE_FORMAT(CURDATE(), '%Y-%m-01') 
                     THEN r.distance ELSE 0 END), 0) as current_month_distance,
    -- Dernière activité
    MAX(r.start_time) as last_run_date,
    u.created_at as registration_date
FROM users u
LEFT JOIN runs r ON u.id = r.user_id
GROUP BY u.id, u.username, u.email, u.created_at;
\end{lstlisting}

\subsubsection{Vue Monthly\_Stats}

\begin{lstlisting}[language=sql]
-- Vue pour les statistiques mensuelles
CREATE VIEW monthly_stats AS
SELECT 
    user_id,
    YEAR(start_time) as year,
    MONTH(start_time) as month,
    COUNT(*) as runs_count,
    SUM(distance) as total_distance,
    SUM(duration) as total_duration,
    AVG(avg_speed) as avg_speed,
    MAX(max_speed) as max_speed,
    SUM(calories) as total_calories,
    AVG(CASE WHEN distance > 0 AND duration > 0 
             THEN (duration / 60) / (distance / 1000) 
             ELSE NULL END) as avg_pace_min_per_km,
    MIN(start_time) as first_run_date,
    MAX(start_time) as last_run_date
FROM runs 
WHERE start_time IS NOT NULL 
GROUP BY user_id, YEAR(start_time), MONTH(start_time)
ORDER BY user_id, year DESC, month DESC;
\end{lstlisting}

\subsection{Triggers et Procédures Stockées}

\subsubsection{Trigger de Validation}

\begin{lstlisting}[language=sql]
-- Trigger pour calculs automatiques
DELIMITER //

CREATE TRIGGER tr_runs_before_insert
BEFORE INSERT ON runs
FOR EACH ROW
BEGIN
    -- Calcul automatique de la durée si manquante
    IF NEW.duration IS NULL AND NEW.start_time IS NOT NULL AND NEW.end_time IS NOT NULL THEN
        SET NEW.duration = TIMESTAMPDIFF(SECOND, NEW.start_time, NEW.end_time);
    END IF;
    
    -- Calcul automatique de la vitesse moyenne
    IF NEW.avg_speed IS NULL AND NEW.distance IS NOT NULL AND NEW.duration IS NOT NULL AND NEW.duration > 0 THEN
        SET NEW.avg_speed = NEW.distance / NEW.duration;
    END IF;
    
    -- Estimation des calories si manquante (formule simplifiée)
    IF NEW.calories IS NULL AND NEW.distance IS NOT NULL THEN
        -- Approximation : 1 km = 60 kcal pour une personne de 70kg
        SET NEW.calories = ROUND((NEW.distance / 1000) * 60);
    END IF;
    
    -- Validation que end_time >= start_time
    IF NEW.end_time IS NOT NULL AND NEW.start_time IS NOT NULL AND NEW.end_time < NEW.start_time THEN
        SIGNAL SQLSTATE '45000' SET MESSAGE_TEXT = 'end_time must be >= start_time';
    END IF;
    
    -- Validation distance réaliste
    IF NEW.distance IS NOT NULL AND NEW.distance > 500000 THEN  -- > 500km
        SIGNAL SQLSTATE '45000' SET MESSAGE_TEXT = 'Distance exceeds realistic maximum (500km)';
    END IF;
    
    -- Validation vitesse réaliste
    IF NEW.max_speed IS NOT NULL AND NEW.max_speed > 20 THEN  -- > 72 km/h
        SIGNAL SQLSTATE '45000' SET MESSAGE_TEXT = 'Speed exceeds realistic maximum for running';
    END IF;
END//

CREATE TRIGGER tr_runs_before_update
BEFORE UPDATE ON runs
FOR EACH ROW
BEGIN
    -- Mêmes validations que pour l'insertion
    IF NEW.duration IS NULL AND NEW.start_time IS NOT NULL AND NEW.end_time IS NOT NULL THEN
        SET NEW.duration = TIMESTAMPDIFF(SECOND, NEW.start_time, NEW.end_time);
    END IF;
    
    IF NEW.avg_speed IS NULL AND NEW.distance IS NOT NULL AND NEW.duration IS NOT NULL AND NEW.duration > 0 THEN
        SET NEW.avg_speed = NEW.distance / NEW.duration;
    END IF;
    
    -- Mise à jour automatique du timestamp
    SET NEW.updated_at = CURRENT_TIMESTAMP;
END//

DELIMITER ;
\end{lstlisting}

\subsubsection{Procédures de Maintenance}

\begin{lstlisting}[language=sql]
DELIMITER //

-- Procédure de nettoyage des anciennes données
CREATE PROCEDURE sp_cleanup_old_data(IN days_retention INT)
BEGIN
    DECLARE done INT DEFAULT FALSE;
    DECLARE deleted_count INT DEFAULT 0;
    
    DECLARE CONTINUE HANDLER FOR SQLEXCEPTION
    BEGIN
        ROLLBACK;
        RESIGNAL;
    END;
    
    START TRANSACTION;
    
    -- Suppression des logs d'admin anciens
    DELETE FROM admin_logs 
    WHERE created_at < DATE_SUB(NOW(), INTERVAL days_retention DAY);
    
    SET deleted_count = ROW_COUNT();
    
    -- Log de l'opération
    INSERT INTO admin_logs (admin_id, action, resource_type, details)
    VALUES (1, 'CLEANUP', 'SYSTEM', 
            CONCAT('Deleted ', deleted_count, ' old admin logs'));
    
    COMMIT;
    
    SELECT CONCAT('Cleanup completed. Deleted ', deleted_count, ' records.') as result;
END//

-- Procédure de calcul des statistiques
CREATE PROCEDURE sp_update_user_statistics(IN target_user_id INT)
BEGIN
    DECLARE total_runs INT DEFAULT 0;
    DECLARE total_distance DECIMAL(10,2) DEFAULT 0;
    DECLARE total_duration INT DEFAULT 0;
    
    -- Calcul des totaux
    SELECT 
        COUNT(*),
        COALESCE(SUM(distance), 0),
        COALESCE(SUM(duration), 0)
    INTO total_runs, total_distance, total_duration
    FROM runs 
    WHERE user_id = target_user_id;
    
    -- Mise à jour d'une table de cache (à créer si nécessaire)
    INSERT INTO user_statistics_cache (
        user_id, total_runs, total_distance, total_duration, last_updated
    ) VALUES (
        target_user_id, total_runs, total_distance, total_duration, NOW()
    ) ON DUPLICATE KEY UPDATE
        total_runs = VALUES(total_runs),
        total_distance = VALUES(total_distance),
        total_duration = VALUES(total_duration),
        last_updated = VALUES(last_updated);
        
    SELECT 'Statistics updated successfully' as result;
END//

-- Procédure de backup des données utilisateur
CREATE PROCEDURE sp_export_user_data(IN target_user_id INT)
BEGIN
    DECLARE user_exists INT DEFAULT 0;
    
    -- Vérifier que l'utilisateur existe
    SELECT COUNT(*) INTO user_exists FROM users WHERE id = target_user_id;
    
    IF user_exists = 0 THEN
        SIGNAL SQLSTATE '45000' SET MESSAGE_TEXT = 'User not found';
    END IF;
    
    -- Export des données utilisateur
    SELECT 'user_profile' as data_type, 
           JSON_OBJECT(
               'id', id,
               'username', username,
               'email', email,
               'first_name', first_name,
               'last_name', last_name,
               'date_of_birth', date_of_birth,
               'height', height,
               'weight', weight,
               'created_at', created_at,
               'updated_at', updated_at
           ) as data
    FROM users WHERE id = target_user_id
    
    UNION ALL
    
    -- Export des courses
    SELECT 'runs' as data_type,
           JSON_ARRAYAGG(
               JSON_OBJECT(
                   'id', id,
                   'start_time', start_time,
                   'end_time', end_time,
                   'duration', duration,
                   'distance', distance,
                   'avg_speed', avg_speed,
                   'max_speed', max_speed,
                   'calories', calories,
                   'route_data', route_data,
                   'created_at', created_at
               )
           ) as data
    FROM runs WHERE user_id = target_user_id;
END//

DELIMITER ;
\end{lstlisting}

\subsection{Index et Optimisations}

\subsubsection{Index Composés Avancés}

\begin{lstlisting}[language=sql]
-- Index optimisés pour les requêtes fréquentes

-- Index pour les requêtes de dashboard (courses récentes par utilisateur)
CREATE INDEX idx_runs_dashboard 
ON runs(user_id, start_time DESC, distance DESC);

-- Index pour les statistiques mensuelles
CREATE INDEX idx_runs_monthly_stats 
ON runs(user_id, start_time, distance, duration, calories);

-- Index pour la recherche par plage de dates
CREATE INDEX idx_runs_date_range 
ON runs(start_time, end_time, user_id);

-- Index pour les requêtes d'administration
CREATE INDEX idx_admin_logs_monitoring 
ON admin_logs(created_at DESC, action, admin_id);

-- Index partiel pour les courses terminées seulement
CREATE INDEX idx_runs_completed 
ON runs(user_id, start_time DESC) 
WHERE end_time IS NOT NULL;

-- Index fonctionnel pour le calcul de l'allure
CREATE INDEX idx_runs_pace 
ON runs(user_id, ((duration / 60) / (distance / 1000))) 
WHERE distance > 0 AND duration > 0;
\end{lstlisting}

\subsubsection{Partitioning Strategy}

\begin{lstlisting}[language=sql]
-- Partitioning de la table runs par année (pour de gros volumes)
ALTER TABLE runs PARTITION BY RANGE (YEAR(start_time)) (
    PARTITION p2022 VALUES LESS THAN (2023),
    PARTITION p2023 VALUES LESS THAN (2024),
    PARTITION p2024 VALUES LESS THAN (2025),
    PARTITION p2025 VALUES LESS THAN (2026),
    PARTITION p2026 VALUES LESS THAN (2027),
    PARTITION p_future VALUES LESS THAN MAXVALUE
);

-- Partitioning des logs par trimestre
ALTER TABLE admin_logs PARTITION BY RANGE (TO_DAYS(created_at)) (
    PARTITION p2024_q1 VALUES LESS THAN (TO_DAYS('2024-04-01')),
    PARTITION p2024_q2 VALUES LESS THAN (TO_DAYS('2024-07-01')),
    PARTITION p2024_q3 VALUES LESS THAN (TO_DAYS('2024-10-01')),
    PARTITION p2024_q4 VALUES LESS THAN (TO_DAYS('2025-01-01')),
    PARTITION p_current VALUES LESS THAN MAXVALUE
);
\end{lstlisting}

\subsection{Requêtes d'Analyse Avancées}

\subsubsection{Analyses de Performance}

\begin{lstlisting}[language=sql]
-- Évolution des performances utilisateur
SELECT 
    user_id,
    DATE(start_time) as run_date,
    distance / 1000 as distance_km,
    duration / 60 as duration_minutes,
    (duration / 60) / (distance / 1000) as pace_min_per_km,
    -- Moyenne mobile sur 5 courses
    AVG((duration / 60) / (distance / 1000)) 
        OVER (PARTITION BY user_id 
              ORDER BY start_time 
              ROWS BETWEEN 4 PRECEDING AND CURRENT ROW) as avg_pace_5_runs,
    -- Comparaison avec la course précédente
    LAG((duration / 60) / (distance / 1000)) 
        OVER (PARTITION BY user_id ORDER BY start_time) as previous_pace,
    -- Rang de la performance
    RANK() OVER (PARTITION BY user_id ORDER BY (duration / 60) / (distance / 1000)) as pace_rank
FROM runs
WHERE distance > 1000 AND duration > 0  -- Minimum 1km
ORDER BY user_id, start_time;

-- Top utilisateurs par distance mensuelle
SELECT 
    u.username,
    u.email,
    YEAR(r.start_time) as year,
    MONTH(r.start_time) as month,
    COUNT(r.id) as runs_count,
    ROUND(SUM(r.distance) / 1000, 2) as total_km,
    ROUND(AVG(r.distance) / 1000, 2) as avg_km_per_run,
    ROUND(MIN((r.duration / 60) / (r.distance / 1000)), 2) as best_pace,
    -- Ranking mensuel
    RANK() OVER (PARTITION BY YEAR(r.start_time), MONTH(r.start_time) 
                 ORDER BY SUM(r.distance) DESC) as monthly_rank
FROM users u
JOIN runs r ON u.id = r.user_id
WHERE r.start_time >= DATE_SUB(CURDATE(), INTERVAL 12 MONTH)
  AND r.distance > 0
GROUP BY u.id, YEAR(r.start_time), MONTH(r.start_time)
HAVING runs_count >= 3  -- Minimum 3 courses dans le mois
ORDER BY year DESC, month DESC, total_km DESC
LIMIT 50;
\end{lstlisting}

\subsubsection{Requêtes de Monitoring}

\begin{lstlisting}[language=sql]
-- Monitoring de l'activité système
SELECT 
    DATE(created_at) as date,
    COUNT(*) as total_runs,
    COUNT(DISTINCT user_id) as active_users,
    ROUND(AVG(distance) / 1000, 2) as avg_distance_km,
    ROUND(AVG(duration) / 60, 2) as avg_duration_minutes,
    -- Détection d'anomalies
    CASE 
        WHEN COUNT(*) > (SELECT AVG(daily_runs) * 2 FROM (
            SELECT DATE(created_at) as d, COUNT(*) as daily_runs 
            FROM runs 
            WHERE created_at >= DATE_SUB(CURDATE(), INTERVAL 30 DAY)
            GROUP BY DATE(created_at)
        ) t) THEN 'HIGH_ACTIVITY'
        WHEN COUNT(*) < (SELECT AVG(daily_runs) * 0.5 FROM (
            SELECT DATE(created_at) as d, COUNT(*) as daily_runs 
            FROM runs 
            WHERE created_at >= DATE_SUB(CURDATE(), INTERVAL 30 DAY)
            GROUP BY DATE(created_at)
        ) t) THEN 'LOW_ACTIVITY'
        ELSE 'NORMAL'
    END as activity_level
FROM runs
WHERE created_at >= DATE_SUB(CURDATE(), INTERVAL 7 DAY)
GROUP BY DATE(created_at)
ORDER BY date DESC;

-- Détection d'utilisateurs suspects (performances irréalistes)
SELECT 
    u.username,
    u.email,
    COUNT(r.id) as total_runs,
    MAX(r.max_speed * 3.6) as max_speed_kmh,  -- Conversion en km/h
    MIN((r.duration / 60) / (r.distance / 1000)) as best_pace_min_km,
    MAX(r.distance / 1000) as longest_run_km,
    'SUSPICIOUS' as flag_reason
FROM users u
JOIN runs r ON u.id = r.user_id
WHERE r.distance > 0 AND r.duration > 0
GROUP BY u.id
HAVING 
    max_speed_kmh > 50 OR          -- Vitesse > 50 km/h (irréaliste)
    best_pace_min_km < 2.5 OR      -- Allure < 2:30 min/km (record mondial)
    longest_run_km > 200           -- Course > 200km
ORDER BY max_speed_kmh DESC;
\end{lstlisting}

\subsection{Scripts de Maintenance}

\subsubsection{Optimisation des Performances}

\begin{lstlisting}[language=sql]
-- Script d'optimisation à exécuter régulièrement

-- Analyse des tables
ANALYZE TABLE users, runs, admin_logs;

-- Optimisation des tables
OPTIMIZE TABLE users, runs, admin_logs;

-- Vérification de l'utilisation des index
SELECT 
    TABLE_NAME,
    INDEX_NAME,
    CARDINALITY,
    CASE 
        WHEN CARDINALITY = 0 THEN 'UNUSED'
        WHEN CARDINALITY < 10 THEN 'LOW_SELECTIVITY'
        ELSE 'GOOD'
    END as index_status
FROM INFORMATION_SCHEMA.STATISTICS 
WHERE TABLE_SCHEMA = 'running_app_db'
  AND TABLE_NAME IN ('users', 'runs', 'admin_logs')
ORDER BY TABLE_NAME, INDEX_NAME;

-- Identification des requêtes lentes
SELECT 
    DIGEST_TEXT,
    COUNT_STAR as execution_count,
    ROUND(AVG_TIMER_WAIT / 1000000000, 2) as avg_duration_seconds,
    ROUND(MAX_TIMER_WAIT / 1000000000, 2) as max_duration_seconds,
    ROUND(SUM_ROWS_EXAMINED / COUNT_STAR, 0) as avg_rows_examined
FROM performance_schema.events_statements_summary_by_digest
WHERE DIGEST_TEXT LIKE '%runs%' OR DIGEST_TEXT LIKE '%users%'
ORDER BY avg_duration_seconds DESC
LIMIT 10;
\end{lstlisting}

\subsubsection{Backup et Archivage}

\begin{lstlisting}[language=sql]
-- Script de préparation pour backup

-- Création d'une table d'archivage pour les anciennes courses
CREATE TABLE runs_archive LIKE runs;

-- Archivage des courses de plus de 2 ans
INSERT INTO runs_archive 
SELECT * FROM runs 
WHERE start_time < DATE_SUB(CURDATE(), INTERVAL 2 YEAR);

-- Vérification de l'intégrité avant suppression
SELECT 
    COUNT(*) as total_to_archive,
    MIN(start_time) as oldest_run,
    MAX(start_time) as newest_in_archive
FROM runs 
WHERE start_time < DATE_SUB(CURDATE(), INTERVAL 2 YEAR);

-- Suppression après confirmation manuelle
-- DELETE FROM runs WHERE start_time < DATE_SUB(CURDATE(), INTERVAL 2 YEAR);

-- Compression de l'archive (commande système)
-- mysqldump running_app_db runs_archive | gzip > runs_archive_$(date +%Y%m%d).sql.gz
\end{lstlisting}