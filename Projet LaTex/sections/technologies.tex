\section{Technologies et Justifications}

\subsection{Stack Technologique Globale}

\begin{figure}[H]
\centering
\begin{tikzpicture}[node distance=1.5cm]
    % Frontend
    \node[rectangle, draw, fill=green!20, text width=3cm, text centered] (rn) {
        \textbf{React Native}\\
        \small Frontend Mobile
    };
    \node[rectangle, draw, fill=green!10, text width=2.5cm, text centered, right of=rn, xshift=1cm] (expo) {
        \textbf{Expo}\\
        \small Toolchain
    };
    
    % Backend
    \node[rectangle, draw, fill=blue!20, text width=3cm, text centered, below of=rn] (flask) {
        \textbf{Flask}\\
        \small API Backend
    };
    \node[rectangle, draw, fill=blue!10, text width=2.5cm, text centered, right of=flask, xshift=1cm] (sql) {
        \textbf{SQLAlchemy}\\
        \small ORM
    };
    
    % Database
    \node[rectangle, draw, fill=orange!20, text width=3cm, text centered, below of=flask] (mysql) {
        \textbf{MySQL}\\
        \small Base de Données
    };
    \node[rectangle, draw, fill=orange!10, text width=2.5cm, text centered, right of=mysql, xshift=1cm] (redis) {
        \textbf{Redis}\\
        \small Cache (futur)
    };
    
    % Arrows
    \draw[->, thick] (rn) -- (flask);
    \draw[->, thick] (flask) -- (mysql);
    \draw[<->, thick] (rn) -- (expo);
    \draw[<->, thick] (flask) -- (sql);
    \draw[-->, dashed] (flask) -- (redis);
    
\end{tikzpicture}
\caption{Stack Technologique Running App}
\end{figure}

\subsection{Frontend Mobile - React Native}

\subsubsection{Choix de React Native}

\paragraph{Justifications Techniques}
\begin{itemize}
    \item \textbf{Performance native} : Rendu optimisé pour iOS et Android
    \item \textbf{Écosystème riche} : Nombreuses bibliothèques spécialisées
    \item \textbf{Hot Reload} : Développement rapide avec rechargement instantané
    \item \textbf{Communauté active} : Support Meta et communauté Open Source
    \item \textbf{Courbe d'apprentissage} : Syntaxe React familière
\end{itemize}

\paragraph{Justifications Business}
\begin{itemize}
    \item \textbf{Time-to-market} : Développement simultané iOS/Android
    \item \textbf{Maintenance} : Une seule codebase à maintenir
    \item \textbf{Équipe} : Compétences JavaScript réutilisables
    \item \textbf{Budget} : Coût de développement réduit vs développement natif
\end{itemize}

\subsubsection{Bibliothèques Principales}

\begin{table}[H]
\centering
\begin{tabular}{|l|l|l|p{5cm}|}
\hline
\textbf{Bibliothèque} & \textbf{Version} & \textbf{Usage} & \textbf{Justification} \\
\hline
React Navigation & 6.x & Navigation & Standard de facto, performance \\
Expo Location & 17.x & GPS/Géoloc & API unifiée, permissions simplifiées \\
React Native Maps & 1.14.x & Cartographie & Intégration Google/Apple Maps \\
AsyncStorage & 1.23.x & Stockage local & Persistance simple et rapide \\
Axios & 1.7.x & HTTP & Intercepteurs, gestion d'erreurs \\
React Hook Form & 7.x & Formulaires & Performance, validation native \\
\hline
\end{tabular}
\caption{Dépendances Frontend Principales}
\end{table}

\subsubsection{Architecture Frontend}

\paragraph{Structure des Dossiers}
\begin{lstlisting}[language=bash]
src/
├── components/          # Composants réutilisables
│   ├── common/         # Composants génériques
│   └── forms/          # Composants de formulaires
├── screens/            # Écrans de l'application
│   ├── auth/          # Authentification
│   └── main/          # Écrans principaux
├── navigation/         # Configuration navigation
├── context/           # Context API (état global)
├── services/          # Services API et utilitaires
├── utils/             # Fonctions utilitaires
└── constants/         # Constantes et configuration
\end{lstlisting}

\paragraph{Gestion d'État}
\begin{description}
    \item[\textbf{Context API}] État global partagé (authentification, courses)
    \item[\textbf{useState/useReducer}] État local des composants
    \item[\textbf{AsyncStorage}] Persistance locale (tokens, données offline)
\end{description}

\subsection{Backend API - Python Flask}

\subsubsection{Choix de Flask}

\paragraph{Avantages de Flask}
\begin{itemize}
    \item \textbf{Minimalisme} : Framework léger et flexible
    \item \textbf{Modularité} : Architecture par blueprints
    \item \textbf{Écosystème Python} : Bibliothèques scientifiques (calculs de performance)
    \item \textbf{Simplicité} : Configuration et déploiement simplifiés
    \item \textbf{Performance} : Optimisé pour les APIs REST
\end{itemize}

\paragraph{Comparaison avec les Alternatives}

\begin{table}[H]
\centering
\begin{tabular}{|l|c|c|c|c|}
\hline
\textbf{Framework} & \textbf{Simplicité} & \textbf{Performance} & \textbf{Écosystème} & \textbf{Choix} \\
\hline
Flask & ✓✓✓ & ✓✓ & ✓✓✓ & \textcolor{green}{\textbf{Retenu}} \\
Django & ✓ & ✓✓ & ✓✓✓ & Trop lourd \\
FastAPI & ✓✓ & ✓✓✓ & ✓✓ & Moins mature \\
Express.js & ✓✓✓ & ✓✓ & ✓✓ & Stack mixte \\
\hline
\end{tabular}
\caption{Comparaison des frameworks backend}
\end{table}

\subsubsection{Extensions Flask Utilisées}

\begin{table}[H]
\centering
\begin{tabular}{|l|l|p{6cm}|}
\hline
\textbf{Extension} & \textbf{Version} & \textbf{Rôle} \\
\hline
Flask-SQLAlchemy & 3.1.x & ORM et gestion base de données \\
Flask-Migrate & 4.0.x & Migrations automatiques du schéma \\
Flask-JWT-Extended & 4.6.x & Authentification JWT sécurisée \\
Flask-CORS & 4.0.x & Gestion des requêtes cross-origin \\
Flask-Limiter & 3.5.x & Rate limiting et protection DDoS \\
Marshmallow & 3.20.x & Sérialisation/validation des données \\
\hline
\end{tabular}
\caption{Extensions Flask Principales}
\end{table}

\subsubsection{Architecture Backend}

\paragraph{Pattern de Organisation}
\begin{lstlisting}[language=python]
app/
├── __init__.py         # Factory pattern, configuration
├── config.py          # Configuration environnements
├── models/            # Modèles SQLAlchemy
│   ├── user.py
│   └── run.py
├── routes/            # Blueprints (contrôleurs)
│   ├── auth.py
│   ├── users.py
│   └── runs.py
├── services/          # Logique métier
│   ├── auth_service.py
│   └── run_service.py
├── utils/             # Utilitaires et helpers
│   ├── decorators.py
│   └── validators.py
└── tests/             # Tests automatisés
    ├── test_auth.py
    └── test_runs.py
\end{lstlisting}

\subsection{Base de Données - MySQL}

\subsubsection{Choix de MySQL}

\paragraph{Critères de Décision}
\begin{itemize}
    \item \textbf{Robustesse} : SGBD mature et éprouvé en production
    \item \textbf{Performance} : Optimisé pour les requêtes relationnelles
    \item \textbf{Support JSON} : Stockage efficace des données GPS
    \item \textbf{Scaling} : Réplication master-slave, clustering
    \item \textbf{Communauté} : Documentation extensive, support commercial
    \item \textbf{Écosystème} : Intégration native avec SQLAlchemy
\end{itemize}

\paragraph{Configuration Optimisée}
\begin{lstlisting}[language=ini]
# my.cnf - Configuration MySQL pour Running App
[mysqld]
innodb_buffer_pool_size = 1G
innodb_log_file_size = 256M
innodb_flush_log_at_trx_commit = 2
query_cache_type = 1
query_cache_size = 128M
max_connections = 200
slow_query_log = 1
long_query_time = 2
\end{lstlisting}

\subsubsection{Alternatives Considérées}

\begin{table}[H]
\centering
\begin{tabular}{|l|c|c|c|c|l|}
\hline
\textbf{SGBD} & \textbf{Performance} & \textbf{JSON} & \textbf{Scaling} & \textbf{Complexité} & \textbf{Décision} \\
\hline
MySQL & ✓✓✓ & ✓✓ & ✓✓✓ & ✓✓ & \textcolor{green}{\textbf{Retenu}} \\
PostgreSQL & ✓✓✓ & ✓✓✓ & ✓✓✓ & ✓ & Alternative valide \\
MongoDB & ✓✓ & ✓✓✓ & ✓✓ & ✓✓✓ & NoSQL non requis \\
SQLite & ✓ & ✓ & ✓ & ✓✓✓ & Dev uniquement \\
\hline
\end{tabular}
\caption{Comparaison des SGBD}
\end{table}

\subsection{Outils de Développement}

\subsubsection{Environnement de Développement}

\begin{description}
    \item[\textbf{IDE}] Visual Studio Code avec extensions React Native et Python
    \item[\textbf{Versioning}] Git avec GitHub pour le hosting et CI/CD
    \item[\textbf{Package Managers}] npm/yarn (JS), pip/pipenv (Python)
    \item[\textbf{Debugging}] Flipper (React Native), Flask debugger
    \item[\textbf{Testing}] Jest (JS), pytest (Python)
\end{description}

\subsubsection{DevOps et Déploiement}

\begin{table}[H]
\centering
\begin{tabular}{|l|l|l|}
\hline
\textbf{Outil} & \textbf{Usage} & \textbf{Avantage} \\
\hline
Docker & Containerisation & Reproductibilité environnements \\
GitHub Actions & CI/CD Pipeline & Intégration native GitHub \\
Expo EAS & Build mobile & Distribution automatisée \\
Heroku/AWS & Hosting API & Scaling automatique \\
CloudFlare & CDN/DNS & Performance globale \\
\hline
\end{tabular}
\caption{Outils DevOps}
\end{table}

\subsection{Choix d'Architecture}

\subsubsection{Architecture Microservices vs Monolithique}

\paragraph{Décision : Monolithe Modulaire}
\begin{itemize}
    \item \textbf{Simplicité} : Une seule application à déployer et maintenir
    \item \textbf{Performance} : Pas de latence réseau entre services
    \item \textbf{Développement} : Setup et debugging simplifiés
    \item \textbf{Évolution} : Possibilité de migration future vers microservices
\end{itemize}

\subsubsection{Patterns Architecturaux Appliqués}

\begin{description}
    \item[\textbf{Repository Pattern}] Séparation logique métier / accès données
    \item[\textbf{Factory Pattern}] Configuration modulaire de Flask
    \item[\textbf{Decorator Pattern}] Authentification et autorisation
    \item[\textbf{Observer Pattern}] Context API pour la réactivité UI
\end{description}

\subsection{Justifications des Choix}

\subsubsection{Critères de Décision}

\begin{enumerate}
    \item \textbf{Performance} : Réactivité de l'interface utilisateur
    \item \textbf{Maintenabilité} : Code propre et architecture modulaire
    \item \textbf{Scalabilité} : Capacité de montée en charge
    \item \textbf{Sécurité} : Protection des données utilisateur
    \item \textbf{Time-to-market} : Rapidité de développement
    \item \textbf{Coût} : Licensing et infrastructure
\end{enumerate}

\subsubsection{Trade-offs Assumés}

\begin{itemize}
    \item \textbf{React Native vs Natif} : Performance légèrement moindre contre développement unifié
    \item \textbf{Monolithe vs Microservices} : Simplicité contre flexibilité architecturale
    \item \textbf{MySQL vs NoSQL} : Cohérence des données contre flexibilité du schéma
    \item \textbf{JWT vs Sessions} : Stateless contre simplicité de révocation
\end{itemize}